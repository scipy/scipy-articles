% based on template:
% https://tex.stackexchange.com/questions/15532/looking-for-cover-letter-template

\documentclass[10pt,stdletter,dateno]{newlfm}

% to adjust vertical spacing properties of
% newlfm class see: https://tex.stackexchange.com/a/65533/43006
%\unprtop{0mm}
\topmarginsize{0mm}
\topmarginskip{0in}
\headermarginsize{0in}
\headermarginskip{0in}
\MinHead{0in}
%\dateskipafter{0pt}
%\dateskipbefore{0pt}
%\sigskipbefore{0pt}
%\sigskipafter{0pt}
\closeskipbefore{1pt}
%\closeskipafter{0pt}
%\addrfromskipbefore{0pt}
\addrfromskipafter{0in}
\addrtoskipbefore{0in}
%\addrtoskipafter{0pt}
\newlfmP{Headlinewd=0pt,Footlinewd=0pt}
\newlfmP{sigsize=10pt}
\bottommarginskip{0pt}
\MinFoot{0pt}

\usepackage{kpfonts}
\usepackage{url}

\widowpenalty=1000
\clubpenalty=1000

\namefrom{Ralf Gommers,
          Matt Haberland,
          and Tyler Reddy}
\addrfrom{%
    \today\\[10pt]
    Ralf Gommers,\\
    Quansight Labs,\\
    The Netherlands,\\
    \texttt{ralf.gommers@gmail.com},\\
    %phone number\\
    \\
    Matt Haberland,\\
    BioResource and Agricultural Engineering,\\
    California Polytechnic State University,\\
    San Luis Obispo, CA, 93407, USA,\\
    \texttt{mhaberla@calpoly.edu},\\
    %phone number\\
    \\
    Tyler Reddy,\\
    CCS-7 Applied Computer Science Group,\\
    Los Alamos National Laboratory,\\
    Los Alamos, NM, 87545, USA,\\
    \texttt{treddy@lanl.gov},\\
    %phone number\\
}

\addrto{%
Dr. Rita Strack,\\
Nature Methods Editorial Office,\\
One New York Plaza Suite 4500,\\
New York, NY 10004,\\
USA}

\greetto{Dear Dr. Strack,}
\closeline{Sincerely,}
\begin{document}
\begin{newlfm}

We are submitting the enclosed manuscript describing a milestone
1.0 release of SciPy for consideration to be published in
\emph{Nature Methods}, as informally discussed.
SciPy is the fundamental scientific 
computing library in the Python programming language. 
Hundreds of experts have contributed to our library, which
has been downloaded millions of times and has over 100,000 dependent
code repositories, including many used in the life sciences.

We have specifically chosen \emph{Nature Methods} because
SciPy lies at the base of the entire ecosystem of scientific methods
described in the journal scope, and because SciPy is effectively an
ever-improving methodological library. High-profile manuscripts citing SciPy 
extend beyond the biosciences to recent reports on gravitational waves and
black holes. Lacking a better alternative, these citations all point
to the SciPy website; instead, we think the package now merits a
formal publication in a highly visible journal. A pre-print will likely
soon be made available on arXiv.

We suggest the following three peer reviewers as suitable candidates,
and we do not have any specific reviewer exclusion requests:

\begin{enumerate}
    \item Alan Edelman, \texttt{edelman@math.mit.edu}, co-creator of Julia
    programming language and Faculty at MIT.
    \item William Stein, \texttt{wstein@gmail.com}, originator and
    lead developer of SageMath.
    \item Kazunori Akiyama, \texttt{kazu@haystack.mit.edu}, first author
    of event horizon imaging paper that cites SciPy.
\end{enumerate}

Thank you for your consideration.  We look forward to
    hearing from you.\vspace*{0.1cm}
\end{newlfm}
\end{document}
