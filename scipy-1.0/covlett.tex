% based on template:
% https://tex.stackexchange.com/questions/15532/looking-for-cover-letter-template

\documentclass[10pt,stdletter,dateno]{newlfm}

% to adjust vertical spacing properties of
% newlfm class see: https://tex.stackexchange.com/a/65533/43006
%\unprtop{0mm}
\topmarginsize{0mm}
\topmarginskip{0in}
\headermarginsize{0in}
\headermarginskip{0in}
\MinHead{0in}
%\dateskipafter{0pt}
%\dateskipbefore{0pt}
\sigskipbefore{40pt}
%\sigskipafter{0pt}
\closeskipbefore{1pt}
%\closeskipafter{0pt}
%\addrfromskipbefore{0pt}
\addrfromskipafter{0in}
\addrtoskipbefore{0in}
%\addrtoskipafter{0pt}
\newlfmP{Headlinewd=0pt,Footlinewd=0pt}
\newlfmP{sigsize=10pt}
\bottommarginskip{10pt}
\MinFoot{0pt}
\leftmarginsize{0.95in}
\rightmarginsize{0.95in}

\usepackage{kpfonts}
\usepackage{hyperref}
\usepackage{url}

\hypersetup{%
  linkbordercolor=blue,% hyperlink borders will be blue
  pdfborderstyle={/S/U/W 1}% border style will be underline of width 1pt
}
\widowpenalty=1000
\clubpenalty=1000

\namefrom{Ralf Gommers,
          Matt Haberland,
          and Tyler Reddy}
\addrfrom{%
    \today\\[10pt]
    Ralf Gommers\\
    Quansight Labs\\
    The Netherlands\\
    \texttt{ralf.gommers@gmail.com}\\
    %phone number\\
    \\
    Matt Haberland\\
    BioResource and Agricultural Engineering\\
    California Polytechnic State University\\
    San Luis Obispo, CA 93407, USA\\
    \texttt{mhaberla@calpoly.edu}\\
    %phone number\\
    \\
    Tyler Reddy\\
    CCS-7 Applied Computer Science Group\\
    Los Alamos National Laboratory\\
    Los Alamos, NM 87545, USA\\
    \texttt{treddy@lanl.gov}\\
    %phone number\\
}

\addrto{%
Dr. Rita Strack\\
Nature Methods Editorial Office\\
One New York Plaza Suite 4500\\
New York, NY 10004, USA}

\greetto{Dear Dr. Strack,}
\closeline{Sincerely,}
\begin{document}
\begin{newlfm}

We are submitting the enclosed manuscript describing a milestone
1.0 release of SciPy for consideration to be published in
\emph{Nature Methods}, as informally discussed.
SciPy is the fundamental scientific 
computing library in the Python programming language. 

Hundreds of experts have contributed to our library, which
has been downloaded millions of times and has over 100,000 dependent
code repositories, including many used in the life sciences.

SciPy lies at the base of the ecosystem of life science computational
methods. Many studies, including an extremely influential \emph{Nature 
Methods} paper (\href{https://www.nature.com/articles/nmeth.f.303}
{Caporaso \emph{et al.}~2010}), describe software that 
includes SciPy as a dependency without citing it. Others in \emph{
Nature Methods} (e.g.,\href{https://www.nature.com/articles/s41592-018-0114-z}{
Gruber \emph{et al.}~2018};\href{https://www.nature.com/articles/s41592-019-0403-1}{
Moen \emph{et al.}~2019}) -- and some 3000+ other articles -- 
cite SciPy's website for lack of
a better alternative. As SciPy is central to so many biological studies,
the research community would benefit from an article about SciPy to reference.
Our preprint \href{https://arxiv.org/abs/1907.10121}{available on arXiv} 
only partially fulfills this need; we appreciate your consideration of
the manuscript for publication in \emph{Nature Methods}.

We suggest the following three peer reviewers as suitable candidates,
and we do not have any specific reviewer exclusion requests:

\begin{enumerate}
    \item Alan Edelman, \texttt{edelman@math.mit.edu}, co-creator of Julia
    programming language and Faculty at MIT.
    \item William Stein, \texttt{wstein@gmail.com}, originator and
    lead developer of SageMath.
    \item Kazunori Akiyama, \texttt{kazu@haystack.mit.edu}, first author
    of event horizon imaging paper that cites SciPy.
\end{enumerate}

\end{newlfm}
\end{document}
