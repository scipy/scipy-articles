\documentclass[fleqn,10pt]{wlscirep}

% Packages
\usepackage[super]{nth}
\usepackage{rotating}
\usepackage{makecell}
\usepackage{pifont}
\usepackage{amsfonts}
\usepackage{amsmath}
\usepackage{float}

% New Commands
\newcommand{\cmark}{\ding{51}}%
\newcommand{\xmark}{\ding{55}}%


\title {SciPy - some descriptive title here}

\author[1]{Pauli Virtanen}
\author[2,*]{Ralf Gommers}
\author[3]{TODO}
\affil[1]{Affiliation, department, city, postcode, country}
\affil[2]{Affiliation, department, city, postcode, country}
\affil[2]{Affiliation, department, city, postcode, country}

\affil[*]{ralf.gommers@gmail.com}

\keywords{Scientific computing, Python, Mathematics}

\begin{abstract}
TODO. Abstract must be under 200 words and not include subheadings or citations.
\end{abstract}
\begin{document}

\flushbottom
\maketitle
\thispagestyle{empty}

\section*{Introduction}

\textit{The Introduction section expands on the background of the work (some overlap with the Abstract is acceptable). The introduction should not include subheadings.}

\textbf{History}

\textbf{Project goals and scope}

\textbf{Current status (maturity, users)}


\section*{Architecture and implementation choices}

\subsection*{Submodule organisation}
The SciPy library is organized as a collection of subpackages.  The 16
subpackages include mathematical building blocks (e.g. linear algebra, Fourier
transforms, special functions), data structures (e.g. sparse matrices, k-D trees),
algorithms (e.g. numerical optimization and integration, clustering, interpolation,
graph algorithms, computational geometry), and higher-level data analysis
functionality (e.g. signal and image processing, statistical methods).

Here we summarize the scope and capabilities of each subpackage.

\begin{description}

\item[\texttt{cluster}] ~ \newline
    The \texttt{cluster} subpackage contains two submodules:
    \texttt{cluster.vq} provides vector quantization and k-means algorithms,
    and \texttt{cluster.hierarchy} provides functions for hierarchical and
    agglomerative clustering.
\item[\texttt{constants}] ~ \newline
    Physical and mathematical constants, including the CODATA recommended
    values of the fundamental physical constants\cite{CODATA2014}.
\item[\texttt{fftpack}] ~ \newline
    Fast Fourier Transform routines.  In addition to the FFT, the subpackage
    includes functions for the discrete sine and cosine transforms and for
    pseudo-differential operators.
\item[\texttt{integrate}] ~ \newline
    The \texttt{integrate} subpackage provides tools for the numerical
    computation of single and multiple definite integrals, and for the
    solution of ordinary differential equations, including initial value
    problems and two-point boundary value problems.
\item[\texttt{interpolate}] ~ \newline
    The \texttt{interpolate} subpackage contains spline functions and
    classes, one-dimensional and multi-dimensional (univariate and
    multivariate) interpolation classes, Lagrange and Taylor polynomial
    interpolators, and wrappers for FITPACK and DFITPACK functions.
\item[\texttt{io}] ~ \newline
    A collection of functions and classes for reading and writing MATLAB, IDL,
    Matrix Market, Fortran, NetCDF, Harwell-Boeing, WAV and ARFF data files. 
\item[\texttt{linalg}] ~ \newline
    Linear algebra functions, including:
    elementary functions of a matrix, such as the trace, determinant, norm and
    condition number;
    basic solver for $Ax = b$;
    specialized solvers for Toeplitz matrices, circulant matrices, triangular
    matrices and other structured matrices; least squares solver and
    pseudo-inverse calculations; eigenvalue and eigenvector calculations
    (basic and generalized); matrix decompositions, including Cholesky, Schur,
    Hessenberg, $LU$, $LDL^{\intercal}$, $QR$, $QZ$, singular value, and polar;
    and functions to create specialized matrices, such as diagonal, Toeplitz,
    Hankel, companion, Hilbert, and more.
\item[\texttt{ndimage}] ~ \newline
    This package contains various functions for multi-dimensional image
    processing, including: convolution and assorted linear and nonlinear
    filters (Gaussian filter, median filter, Sobel filter, etc.);
    interpolation; region labeling and processing; and mathematical morphology
    functions.
\item[\texttt{misc}] ~ \newline
    A collection of functions that did not fit into the other subpackages.
    While this subpackage still exists in version 1.0.0, effort is continuing
    to deprecate the contents of this subpackage and eventually remove it.
\item[\texttt{odr}] ~ \newline
    Orthogonal distance regression, including Python wrappers for the Fortran
    library ODRPACK.
\item[\texttt{optimize}] ~ \newline
    The package includes the following, with additional details in the SI:
    implementations of many minimization algorithms; a general linear
    programming solver; a routine for least-squares curve fitting; and a
    collection of general nonlinear solvers for root-finding.
\item[\texttt{signal}] ~ \newline
    The \texttt{signal} subpackage focuses on signal processing (plus some
    basic linear systems theory).  Functionality includes:
    convolution and correlation; splines; filtering and filter design;
    continuous and discrete time linear systems; waveform generation;
    window functions; wavelet computations; peak finding; and spectral
    analysis.  
\item[\texttt{sparse}] ~ \newline
    This package includes implementations of several representations of
    sparse matrices.  It contains two subpackages, 
    \texttt{scipy.sparse.linalg} and \texttt{scipy.sparse.csgraph}.

    \texttt{scipy.sparse.linalg} provides a collection of linear algebra
    routines that work with sparse matrices, including linear equation
    solvers, eigenvalue decomposition, singular value decomposition
    and LU factorization.

    \texttt{scipy.sparse.csgraph} provides a collections of graph algorithms
    for which the graph is represented using a sparse matrix.  Algorithms
    include connected components, shortest path, minimum spanning tree
    and more.
\item[\texttt{spatial}] ~ \newline
    This subpackage provides spatial data structures and algorithms,
    including the k-d tree, Delaunay triangulation, convex hulls and Voronoi
    diagrams.  The subpackage \texttt{scipy.spatial.distance} provides
    a large collection of distance functions, along with functions for
    computing the distance between all pairs of vectors in a given collection
    of points or between all pairs from two collections of points.
\item[\texttt{special}] ~ \newline
    The name comes the class of functions traditionally known as \emph{special
    functions}, but over time, the module has grown to include functions
    beyond the classical special functions.  A more appropriate characterization
    of this subpackage is simply \emph{useful functions}.
    It includes: a large collection of the classical special functions
    such as Airy, Bessel, etc.; orthogonal families of polynomials;
    the Gamma function, and functions related to it;
    functions for computing the PDF, CDF and quantile function for several
    probability distributions;
    information theory functions;
    combinatorial functions \texttt{comb} and \texttt{factorial};
    and more.
\item[\texttt{stats}] ~ \newline
    The \texttt{stats} subpackage provides: a large collection of continuous
    and discrete \emph{probability distributions}, each with methods to compute
    the PDF or PMF, CDF, moments and other statistics, generation of random
    variates, and more;
    \emph{statistical tests}, including Pearson's correlation, Spearman's rank-order
    correlations, Kendall's tau, chi-squared test and its generalization as the
    Cressie-Read power divergence, contingency table tests including Fisher's
    exact test and Mood's median test, and many more;
    and assorted \emph{transformations and statistics} of data.
\end{description}


\subsection*{Common infrastructure}

\subsection*{Language preferences}

\subsection*{API and ABI evolution}


\section*{Key technical improvements}

Here we describe key technical improvements made in the last three years.

\subsection*{Data structures}
\textbf{cKDTree}
\textbf{Sparse matrices}

\subsection*{Unified bindings to compiled code}
LowLevelCallable

\subsection*{Cython bindings for BLAS, LAPACK and special}

\subsection*{Numerical optimization}
\newcommand{\RR}{\ensuremath{\mathbb{R}}}
The \texttt{scipy.optimize} subpackage provides functions for the numerical
solution of several classes of root finding and optimization problems.
Here we highlight recent additions through SciPy 1.0.

%\subsubsection{Root Finding}
%The general ``root finding'' problem is to find a root $\mathbf{x} \in \RR^m$ of $\mathbf{f}: \RR^m \rightarrow \RR^m$, that is, to solve
%\begin{equation}
%\mathbf{f}(\mathbf{x}) = \mathbf{0}
%\end{equation}
%for a solution $\mathbf{x}$.\footnote{Equivalently the problem is to simultaneously find the roots $x_i \in \RR$ of several scalar functions $f_i : \RR \rightarrow \RR$, that is, to solve $f_i(x_0, x_1, \dots, x_{m-1}) = 0$ for $x_i$, $i \in \{0, 1, \dots {m-1}\}$.} The function \texttt{scipy.optimize.root} provides a common interface to several algorithms for solving problems of this type. For the special case\footnote{that is, to solve a single scalar equation $f(x) = 0$ for a single scalar variable $x$} $m = 1$, any one of several specialized functions \texttt{brentq}, \texttt{brenth}, \texttt{ridder}, \texttt{bisect}, or \texttt{newton} may provide improved performance or accuracy. (Have there been any recent improvements? Do we want to summarize the methods as @antonior92 has done for $minimize$? Do we have to explain that these methods only provide \emph{one} solution, and that they are iterative based on a user-provided guess? Do we have to explain the notion of tolerance? Is this a good template for the beginning of the following subsections?)


\subsubsection*{Linear Optimization}

A new interior-point optimizer for continuous linear programming problems, \texttt{linprog} with \texttt{method='interior-point'}, was released with SciPy 1.0. Implementing the core algorithm of the commercial solver MOSEK \cite{andersen2000mosek}, it solves all of the 90+ NETLIB LP benchmark problems \cite{netlib} tested. Unlike some interior point methods, this homogeneous self-dual formulation provides certificates of infeasibility or unboundedness as appropriate. 

A presolve routine \cite{andersen1995presolving} solves trivial problems and otherwise performs problem simplifications, such as bound tightening and removal of fixed variables, and one of several routines for eliminating redundant equality constraints is automatically chosen to reduce the chance of numerical difficulties caused by singular matrices. Although the main solver implementation is pure Python, end-to-end sparse matrix support and heavy use of SciPy's compiled linear system solvers---often for the same system with multiple right hand sides due to the predictor-corrector approach---provide speed sufficient for problems with tens of thousands of variables and constraints.

Compared to the previously implemented simplex method, the new interior-point method is faster for all but the smallest problems, and is suitable for solving medium- and large-sized problems on which the existing simplex implementation fails. However, the interior point method typically returns a solution near the center of an optimal face, yet basic solutions are often preferred for sensitivity analysis and for use in mixed integer programming algorithms. This motivates the need for a crossover routine or a new implementation of the simplex method for sparse problems in a future release, either of which would require an improved sparse linear system solver with efficient support for rank-one updates.

\subsubsection*{Nonlinear Optimization}
\paragraph{Local Minimization}
The \texttt{minimize} function provides a unified interface for finding local minima of nonlinear optimization problems. Four new methods for unconstrained optimization were added to \texttt{minimize} in recent versions of SciPy: \texttt{dogleg}, \texttt{trust-ncg}, \texttt{trust-exact}, and \texttt{trust-krylov}. All are trust-region methods that build a local model of the objective function based on first and second derivative information, approximate the best point within a local ``trust region'', and iterate until a local minimum of the original objective function is reached, but each has unique characteristics that make it appropriate for certain types of problems. For instance, \texttt{trust-exact} achieves fast convergence by solving the trust-region subproblem almost exactly, but it requires the second derivative Hessian matrix to be stored and factored every iteration, which may preclude the solution of large problems ($\geq 1000$ variables). On the other hand, \texttt{trust-ncg} and \texttt{trust-krylov} are well-suited to large-scale optimization problems because they do not need to store and factor the Hessian explicitly, instead using second derivative information in a faster, approximate way. A detailed comparison of the characteristics of all \texttt{minimize} methods is presented in Table~\ref{tab:minimize-si}; it illustrates the level of completeness that SciPy aims for when covering a numerical method or topic.

\paragraph{Global Minimization}
As \texttt{minimize} may return any local minimum, some problems require the use of a global optimization routine. The new \texttt{scipy.optimize.differential\textunderscore evolution} function \cite{Wormington1999,Storn1997} is a stochastic global optimizer that works by evolving a population of candidate solutions. In each iteration, trial candidates are generated by combination of candidates from the existing population. If the trial candidates represent an improvement, then the population is updated. Most recently, the SciPy benchmark suite gained a comprehensive set of 196 global optimization problems for tracking the performance of existing solvers over time and for evaluating whether the performance of new solvers merits their inclusion in the package.

\setlength{\tabcolsep}{3pt}
\begin{table}[H]
  \centering
  \caption{Optimization methods from \texttt{minimize}, which solves problems of the form $\min_x f(x)$, where $x \in \mathbb{R}^n$ and $f: \mathbb{R}^n \rightarrow \mathbb{R}$ .  The field \textit{version added} specifies the algorithm's first appearance in SciPy. Algorithms with \textit{version added} ``0.6*'' were added in version 0.6 or before.
    The field \textit{wrapper} indicates whether the implementation available in SciPy wraps a function written in a compiled language
    (e.g. C or FORTRAN). The fields \textit{\nth{1}} and \textit{\nth{2} derivatives}
    indicates whether first or second order derivatives are required. When \textit{\nth{2} derivatives} is flagged
    with $\sim$ the algorithm does not requires second-order derivatives from
    the user; it computes an approximation internally and uses it to accelerate method convergence.
    \textit{Iterative Hessian factorization} denotes algorithms that factorize the Hessian in an iterative way,
    which does not require explicit matrix factorization or storage of the Hessian.
    \textit{Local convergence} gives a lower bound on the rate of convergence of the iterations sequence once the
    iterate is sufficiently close to the solution: linear (L), superlinear (S) and quadratic (Q). Convergence rates denoted S$^*$ indicate that the algorithm
    has a superlinear rate for the parameters used in SciPy, but can  achieve a quadratic convergence rate with other parameter choices.
    \textit{Global convergence} is marked for the algorithms with guarantees of convergence to a stationary
    point (i.e. a point $x^*$ for which $\nabla f(x^*) = 0$); this is \emph{not} a guarantee of convergence to a global minimum. The table also indicates which algorithms
    can deal with constraints on the variables. We distinguish among \textit{bound constraints} (i.e. $x^l \le x \le x^u$),
    \textit{equality constraints} (i.e. $c_{\text{eq}}(x) = 0$) and \textit{inequality constraints} (i.e. $c_{\text{ineq}}(x) \ge 0$).}
  \begin{tabular}{cccccccccccccc}
      & \rotatebox{80}{\texttt{Nelder-Mead}} & \rotatebox{80}{\texttt{Powell}} & \rotatebox{80}{\texttt{COBYLA}} & \rotatebox{80}{\texttt{CG}} & \rotatebox{80}{\texttt{BFGS}}&  \rotatebox{80}{\texttt{L-BFGS-B}} & \rotatebox{80}{\texttt{SLSQP}} & \rotatebox{80}{\texttt{TNC}} & \rotatebox{80}{\texttt{Newton-CG}} & \rotatebox{80}{\texttt{dogleg}} & \rotatebox{80}{\texttt{trust-ncg}} & \rotatebox{80}{\texttt{trust-exact}} & \rotatebox{80}{\texttt{trust-krylov}} \\
    \hline
    Version added &  0.6* &  0.6* &  0.6* &  0.6* &  0.6* &  0.6* &  0.9 &  0.6* &  0.6* & 0.13 & 0.13 & 0.19 & 1.0 \\
    \hline
    Wrapper & & & \cmark & & & \cmark & \cmark & \cmark & &  & & & \cmark \\
    \hline
    \nth{1} derivatives &  & & & \cmark  & \cmark & \cmark & \cmark & \cmark & \cmark & \cmark & \cmark & \cmark & \cmark \\
    \hline
    \nth{2} derivatives &  &  &  &  & $\sim$ & $\sim$ & $\sim$ & \cmark & \cmark & \cmark & \cmark & \cmark & \cmark \\
    \hline
    \makecell{Iterative Hessian \\
    factorization} & & & &  & & & & \cmark & \cmark &  & \cmark &  & \cmark \\
    \hline
    Local convergence& & & & L & S &  L & S & S$^*$ & S$^*$ & Q & S$^*$ & Q & S$^*$  \\
    \hline
    Global convergence & & &  &   & \cmark & \cmark & \cmark & \cmark & \cmark & \cmark & \cmark & \cmark & \cmark  \\
    \hline
    \makecell{Line-search (LS) or\\ trust-region (TR)} & Neither  & LS &  TR & LS & LS & LS & LS & LS & LS & TR & TR & TR & TR \\
    \hline
    Bound constraints &&&\cmark&&&&\cmark&\cmark&\cmark&&&& \\
    \hline
    Equality constraints &&&&&&&\cmark&&&&& \\
    \hline
    Inequality constraint &&&\cmark&&&&\cmark&&&&& \\
    \hline
    References & \cite{nelder_simplex_1965, wright_direct_1996} & \cite{powell_efficient_1964} &
      \cite{powell_direct_1994, powell_direct_1998, powell_view_2007} &
      \cite{polak_note_1969, nocedal_numerical_2006} & \cite{nocedal_numerical_2006} & \cite{byrd_limited_1995, zhu_algorithm_1997} &
      \cite{schittkowski_nonlinear_1982, schittkowski_nonlinear_1982-1, schittkowski_convergence_1983, kraft_software_1988} &
      \cite{nash_newton-type_1984} & \cite{nocedal_numerical_2006}  & 
      \cite{powell_new_1970, nocedal_numerical_2006} &  \cite{steihaug_conjugate_1983, nocedal_numerical_2006} &
      \cite{conn_trust_2000, more_computing_1983} & \cite{gould_solving_1999, lenders_trlib:_2016} \\
    \hline
  \end{tabular}
  \label{tab:minimize-si}
\end{table}





\subsection*{Statistical distributions}

\subsection*{Polynomial interpolators}

\subsection*{Test and benchmark suite}

    \subsubsection*{Benchmark suite}
    The airspeed velocity (asv) library enables benchmarking Python packages over their lifetimes, and the performance of the SciPy
    code base was monitored with asv starting in February of 2015 (PR \#4501). In addition to ensuring that unit tests are passing (see above),
    confirming that performance generally remains constant or improves over the commit hash history of the project allows us to objectively
    measure that our code base is improving, to empower scientific applications.

    Consider the following asv benchmark results, spanning roughly nine years of project history, that demonstrate the gradual performance
    improvements in \texttt{scipy.spatial.cKDTree.query()} (nearest-neighbor search) performance, using command:

    \texttt{python run.py run -e -s 800 --bench "\textbackslash btime\_query\textbackslash b" "02de46a546..b3ddb2c"}

    \begin{figure}[H]
        \centering
        \includegraphics[width=\textwidth]{static/asv_time_query_ckdtree}
        \caption{Airspeed velocity benchmarks for scipy.spatial.cKDTree.query() over a roughly nine year commit history time frame. The results are based on Python 2.7 performance on the master branch of the project using numpy 1.8.2 and Cython versions 0.27.3 and 0.21.1 (for improved backward compatibility). Only the L2 (Euclidean) norm is shown here, and to improve backward compatibility / sampling of the benchmarks there was no application of toroidal topology to the KDTree (boxsize argument was ignored).}
    \end{figure}



\section*{Project organisation and community}

\textbf{Governance}

\textbf{Roadmap}

\textbf{Community beyond the SciPy library}

\textbf{Maintainers and contributors}


\section*{Discussion}

\textit{The Discussion should be succinct and must not contain subheadings.}

\textbf{Impact now}

\textbf{Future development}.
\textit{This section should include key issues: sparse arrays, ndimage pixel vs point, splines, fftpack vs. np.fft and linalg vs. np.linalg, under-maintained submodules.}


\bibliography{references}

\noindent Use the cite command for an inline citation, e.g. \cite{behnel2011cython}.

\section*{Acknowledgements (not compulsory)}

Acknowledgements should be brief, and should not include thanks to anonymous referees and editors, or effusive comments. Grant or contribution numbers may be acknowledged.

\section*{Author contributions statement}

Must include all authors, identified by initials, for example:
A.A. conceived the experiment(s),  A.A. and B.A. conducted the experiment(s), C.A. and D.A. analysed the results.  All authors reviewed the manuscript.

\section*{Additional information}

To include, in this order: \textbf{Accession codes} (where applicable); \textbf{Competing financial interests} (mandatory statement).

The corresponding author is responsible for submitting a \href{http://www.nature.com/srep/policies/index.html#competing}{competing financial interests statement} on behalf of all authors of the paper. This statement must be included in the submitted article file.

\end{document}

%%% Local Variables:
%%% mode: latex
%%% TeX-master: t
%%% End:
